

I will first establish the fact of evolution's action upon the landscape of thermodynamically stable foldings. To do so, I will examine RNA structures for which a single folding pattern is presumed to be functional and misfoldings are physiolgically ineffective.

For strands folded in equilibrium, the probabilty of correct folding, $P_f$ is related to the difference between gibbs free energy of folding (FEF) and misfolding: $\Delta\Delta G$. For a larger ensemble of competing structures in equilibrium, this likelihood of correct folding must be computed via a partition function over suboptimal foldings. I will compute $P_f$ by computing a partition function over RNAfold suboptima. In order to ask whether evolution has selected strand RNAs with high $P_f$, I will compare realized $P_f$ to values for randomly mutated alternate sequences folding to the same structure with similar FEF.

I will first analyze B. subtilis trp-tRNA because \textit{(1)} proper ``cloverleaf'' folding is crucial to tRNA function in translation, \textit{(2)} thermodynamic suboptimals are known to exist \cite{RefWorks:18}, and \textit{(3)} because B. subtilis trp-tRNA binds a riboswitch to sterically regulate transcription antitermination\cite{RefWorks:20}. If evolution has chosen a high $P_f$ sequence for trp-tRNA, I will test signal robustness by asking whether it has done the same for the rest of B. subtilis tRNAs. I will test folding consistency by comparison with XRC tRNA structures. 

Moving beyond bacteria, I will ask whether evolution has favored optimal folding in human miRNA mir-1  where a substantial fraction of the primary sequence is known to have been preserved for reasons other than folding. Asking whether evolution has maximized $P_f$ over the portion of the sequence for which mir-1 has been allowed to vary, I will characterize evolution under constraint. As above, I will test robustness by repeating my analysis over similiar miRNAs and folding consistency compared to XRC.

Next, I will search for evolutionary signatures indicating that alternate structures have been conserved by selection. Using RNAfold I will generate a list of candidates for functional alternate foldings; for each candidate I will evaluate the likelihood that evolution has preserved a folding using a SCFG algorithm similar to EvoFold \cite{RefWorks:21} and corroborating predictions  with a hybrid thermodynamic/evolutionary method\cite{RefWorks:16}.

I will look first at the tBox riboswitch mentioned in tRNA mediated antitermination above. Knowing that the transcribed RNA switches between two foldings in its regulatory role, I will show that evolutionary has preserved base pairing and stacking both in terminator and antiterminator states. I will perform similar analysis on the ``hammerhead ribozyme''\cite{RefWorks:18} for which NMR methods have unveiled microsecond scale dynamics and identify evolutionary signatures for its reacting states. Next, I will turn to a medically interesting case: examining the 335 nt conserved 5' utr leader region of HIV-1, I will ask whether I can establish evolutionary signatures corroborating recent oligonucleotide evidence\cite{RefWorks:19} that HIV mRNA folds alternately into a branched hairpin and long distance interaction structure in order sequester a start codon and regulate the onset of translation.
Finally, I will lay the groundwork for a large scale database search for natural selection of alternate foldings. Automating the methodology above, I intend to probe sequenced RNAs for unexpected function. By structure homology, I will interpret results as dynamic conformations such as ribozymes and riboswitches when possible and suggest that novel mechanisms may be at work when selected alternate foldings are exotic.